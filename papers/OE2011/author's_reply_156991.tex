\documentclass[a4paper,11pt]{letter}
\usepackage{amsmath}
\usepackage{amssymb}
\usepackage{amsthm}
\usepackage{geometry}
\usepackage{upgreek}
\usepackage{siunitx}

\newcommand{\mum}{\ensuremath{\upmu \mbox{m}}}
\newcommand{\nm}{\ensuremath{\mbox{nm}}}

\geometry{a4paper,textwidth=15cm,textheight=20cm}
\def\baselinestretch{1.2}
\signature{Ni Chen, Jiwoon Yeom, Jae-Hyun Jung, Jae-Hyeung Park, and Byoungho Lee}
\pagestyle{empty}

\begin{document}
\begin{letter}{Optics Express editorial office}

\opening{Dear Editor,}
\begin{center}
{\bf \underline{Author's reply to the reviewers' comments} } \\
\end{center}
(``Resolution enhancement of hologram generated based on integral imaging using a hexagonal lens array: 156991)

Authors appreciate reviewer’s constructive comments. We revised the manuscript following the suggestions. We also corrected the style following the guideline. (The original comments of the reviewers are written in italic).

{\noindent {\bf Comments of Reviewer 1:}}

\textit{Overall comment: \\
The manuscript is clearly written, and the work performed is pertinent in the field of holographic IP imaging. The confrontation of theory, simulation, and experiment makes this manuscript very relevant. However there are several points I would like to see addressed before publications.\\
Originality of the work\\
Authors refer to Shaked et al (ref.1) as the source for digital hologram synthesis of 3D objects under incoherent illumination. Though relevant, the original articles are from R.V. Pole, in APL 10, (1967) considering the technique; and from T. Mishina et al. in Applied Optics 45, (2006) for the computation. In this latter article, authors describe more than the method, the article also includes a full description of the aliasing affect due to the sampling of the image. Something very similar to what it is discussed in the present manuscript.\\
Also, the authors cite Petersen and Middleton for the theory of signal sampling that demonstrated an hexagonal array is more efficient than a rectangular one. 
They then state that: our procedure is not a simple extension of the rectangular lens array case as will be shown in Section 2.  Actually, I don't see any difference with the development followed by Petersen and Middleton fifty years ago.}

\begin{enumerate}

\item \textit{I would like the authors bring more emphasis on their personal contribution and the originality of the work considering those articles. A discussion of the aliasing observed in the present experiment and the one described by T. Mishina et al. should be added.}

\bigskip
\textbf{Answer:}
(a) Reference
We have added R.V. Pole, in APL 10, (1967) and T. Mishina et al. in Applied Optics 45, (2006) in the reference. \\

(b) Difference from T. Mishina et al.’s work
As the reviewer pointed out, T Mishina et al. has analyzed the aliasing effect in the hologram generation using integral imaging. However, their analysis cannot be applied to our case due to the difference in the hologram generation method. In their method, a hologram plane is set to some distance from the lens array plane. Then the light propagation from the elemental images to the hologram plane through the lens array is simulated so that the light field in the hologram plane can be calculated. Therefore in essence, the hologram generated by their method reconstructs the light field from the integral imaging display system. On the contrary, our method first generates a number of the orthographic images from the elemental images. Using these orthographic images, the Fourier hologram is calculated. It was shown that the hologram generated by this method is not the emulation of the integral imaging display system but the exact Fourier hologram of the captured 3D object [2]. \\

Due to the difference, the reason for the aliasing is also different. In T.Misina et al.’s work, the aliasing happens when the angle between the light from the elemental images and the reference light is too large such that the interference pattern cannot be sampled properly by the hologram pixel. In our method, the aliasing happens in the process of orthographic image synthesis. Since the orthographic image is sampled at the elemental lens centers, the type of the lens array can affect greatly the aliasing performance. \\

In order to emphasize the originality, we revised the last paragraph of introduction section as followings.\\
“…following the method presented in Ref. [2].” was added to clarify the integral imaging based hologram generation method we used.

“Although there have been usages of hexagonal lens array in integral imaging, there has been no study on its usage for hologram generation based on integral imaging. Moreover the procedure is not a simple extension of the rectangular lens array case as will be shown in Section 2.” 
is modified to:\\
``Although there have been usages of hexagonal lens array in integral imaging, there has been no study on its usage for hologram generation based on integral imaging. Also, although T. Mishina et al. [?] reported an analysis on the aliasing in the hologram generated using integral imaging, their hologram generation method is different from the method considered in this paper and hence their analysis cannot be applied. To the authors’ best knowledge, this is the first report that provides the quantitative analysis and experimental verification on the resolution enhancement of the hologram synthesis using hexagonal lens array.''\\

We also added a discussion on the aliasing.\\

Above Fig. 5, following sentences are added.\\
“Then in the case of no aliasing (due to sufficiently small elemental lens pitch p), the maximum spatial frequency range that can be reconstructed is slightly larger in the case of rectangular lens array. However, in case of aliasing as shown in Fig. 5, the hexagonal shape spatial frequency range reduces the amount of the aliasing with the same elemental lens pitch p and the object bandwidth, presenting better image quality than the rectangular lens array.”\\

In the discussion of simulation result (above Table 3), following sentences are added. 
“Aliasing happens in both cases since the elemental lens pitch p is not sufficiently small in comparison with the bandwidth of the object. As expected, due to larger maximum radial spatial frequency and the hexagonal shape of the cut-off spatial frequency region, it can be seen that there are more aliasing in the rectangular lens array case compared to the hexagonal lens array case.”

\item \textit{Image quality metric \\
Authors calculate the peak signal-to-noise ratio and the normalized cross correlation to define the image quality. A much better approach will be to compare the point spread functions or its cross section.
}

\bigskip
\textbf{Answer:}
As the reviewer pointed out, the point spread function is much more general measure for the image quality in the imaging optics. But the hologram synthesis process using integral imaging is not a direct imaging optical process but contains a number of digital signal processing, which makes it not easy to find a point spread function for the process. Hence we used PSNR and NCC instead of point spread function for the image quality metric. They do not give full information on the overall imaging performance of the system. But at least for specific resultant images, they can give quantitative values for the image quality.

\item \textit{Rhetoric  \\
Considering the hexagonal array has already been use in IP, and the theory behind the image improvement compare to rectangular array is well known, authors cannot claim to "propose a method for enhancing the hologram resolution". Those claims should be rephrased through all the manuscript (including title and abstract) as a "comparison between hexagonal and rectangular arrays".}

\bigskip
\textbf{Answer:} 
Modified according to the guideline.\\
The title is modified from “Resolution enhancement of hologram generated based on integral imaging using a hexagonal lens array” to “Resolution comparison between integral imaging based hologram synthesis methods using rectangular and hexagonal lens arrays”.
We also revised the abstract and last paragraph of introduction accordingly.


\item \textit{Minor changes  \\
Page 2: "circularly band limited signals are sampled 13.4}

\bigskip
\textbf{Answer:"circularly band limited signals are sampled 13.4} 

\item \textit{Page 2: "rectangular lens array case as will be shown in Section 2. In Section 2 we analyze ". Change to "rectangular lens array case as will be shown in Section 2, where we analyze ". }

\bigskip
\textbf{Answer: Modified as recommended.} 

\item \textit{Page 3 Table 1: "L focal length", should be "l".}

\bigskip
\textbf{Answer: “L” is modified as “l”} 

\item \textit{Fig.4: a and b are indeed identical. Only one of the two graphs should be presented.}

\bigskip
\textbf{Answer: We deleted one of them as recommended.} 

\item \textit{Fig 6: blue and red points do not show well in gray scale media. Use open and filled symbols instead. Change should be reflected in the text.}

\bigskip
\textbf{Answer: The blue points are modified to filled circles and the red ones to open circles.} 
\item \textit{Table 2, 3 and 4: figures presented in those tables are reproduced in the text. Author should just refer to the tables and remove the concerned paragraphs.}

\bigskip
\textbf{Answer:} 
The below paragraphs are deleted:\\
1. In the paragraph below table 2, sentences of “With the same capturing system, the number of elemental images is $100(V)\times 80(H)$ in rectangular sampling and $116(V)\times 80(H)$ in hexagonal sampling. The sampling interval in the vertical direction $\Delta y_p$ is 1 mm in the rectangular sampling and about 0.87 mm in the hexagonal sampling.” \\

2. In the paragraph above table 3, “the PSNRs of the reconstructed images are 21.83dB for the conventional one and 24.25dB for the proposed method. The NCC is 0.9649 for conventional method and 0.9758 for proposed method.” \\

3. In the paragraph above table 4, “The average PSNR is 22.48 dB for conventional reconstruction and 26.42 dB for proposed reconstruction.” And “The average NCC is 0.9623 for conventional method and 0.9853 for the proposed method.” 

(10) Figure 10 is labeled figure 101 \\
Answer: Figure ‘101’ is modified as figure ‘10’.

\end{enumerate}

% -------------------------------------------------------------------

{\noindent {\bf Reviewer 2}}
\textit{This paper describes the enhancement of resolution when a hexagon-grid lens array is used instead of rectangle-grid lens array. I recommend the following modifications before publishing, but they are not mandatory.}

\begin{enumerate}

\item \textit{At Table 1: Parameter “L”, Is it Parameter “l”?}

\bigskip
\textbf{Answer:}
“L” is modified as “$l$”

\item \textit{At the bottom of page 4: Suppose the original … and bandwidth $2B_x x 2B_y$ … \\
According to the search function of Adobe Acrobat, there are no other places where $B_x$ or $B_y$ is used, which means you do not let readers to suppose $B_x$ and $2B_y$. Please check other notes as well as $B_x$ and $B_y$}

\bigskip
\textbf{Answer:}
“Suppose the original object has a limited size $2L_x\times 2L_y$ and a bandwidth $2B_x\times 2B_y$.” is changed to “Suppose the original object has a limited size $2L_x\times 2L_y$.”\\

\item \textit{At the top of page 5, The repetition period is given by $\lambda f / \delta u=\lambda/(2 \delta \theta)$ along. It is hard for me to understand the meaning of this mathematical expression without description. I guess it is hard for other readers, too. I recommend mentioning how to derive it or add a reference paper. } 

\bigskip
\textbf{Answer:} 
$\lambda f/\Delta u=\lambda/(2\Delta \theta)$ and $\lambda f/\Delta v=\lambda/(2\Delta \Phi)$ are changed to $\lambda f/\Delta u=\lambda f/M\Delta s=\lambda f/2f\Delta\theta=\lambda/(2\Delta \theta)$ and $\lambda f/\Delta v=\lambda f/M\Delta t=\lambda f/2f\Delta\Phi=\lambda/(2\Delta\Phi)$ respectively. And a reference[17] is added.\\

\item \textit{At Figure 4, I guess (a) and (b) are the same figures. Even if they are not same, it is hard for readers to find the difference between (a) and (b). I recommend modifying the figures.}

\bigskip
\textbf{Answer:} 
Actually they are the same figure. We deleted one of them as recommended.

\item \textit{At Figure 10, Fig.101-> Fig.10}

\bigskip
\textbf{Answer:} 
Figure '101 is corrected to figure `10`.

\item \textit{At Figure 10 (a) to (d), I recommend adding the size of hologram. Is it $2L_u\times 2L_v$? This information helps readers to understand the meaning of these figures.}

\bigskip
\textbf{Answer:} 
The size of the hologram is added to the figure.


\item \textit{At Figure 10, I have no idea why the almost all amplitude is centered in (a) and (c). I recommend mentioning the reason in Section 3.}

\bigskip
\textbf{Answer:} 
The reason why it seems that all the amplitude is centered in the hologram comes from the high contrast of the hologram. We modified the figure with low contrast image by normalizing the hologram. We also added a sentence “Note that in Fig. 10(a) and (c), the contrast was adjusted for better visibility.” above Fig. 10.

\end{enumerate}

{\noindent {\bf Style corrections:}}
\begin{enumerate}

\item \textit{[x] Author Affiliation \\
* Insert a comma after the second to last authors name in the author list.}

\bigskip
\textbf{Answer:}
Modified according to the guideline.


\item \textit{Main text
* Close the space between the figure number and the figure letter when you reference a part of a figure in the text: Fig. 1(b), not Fig. 1 (b)}

\bigskip
\textbf{Answer:}
Modified according to the guideline.

\item \textit{Please ensure that all references, figures and tables are called out in the text. Please also check the ordering of your reference, figure and table call outs. If they are out of order it could cause delays with publishing your article.}

\bigskip
\textbf{Answer:}
Checked according to the guideline.

\end{enumerate}

THANK YOU.

\vskip 2mm

\closing{Sincerely,}

\end{letter}
\end{document}
